%!TEX root = ../article.tex

% Introduction
\section{Introduction}
\label{sec:introduction}
% Growing use of mobile devices
% Multiple sensors
% Apps can adapt according to context
Nowadays, mobile devices are equipped with many sensors such as light sensor, \gls{GPS} and accelerometer.
Also, these devices have access to multiple data sources such as
the user's activity on social networks and calendar.
The applications (apps), that the user can install, have access to this data provided by these sensors and data sources.
Using this data, the apps can adjust settings and allow users to perform tasks according to it.
The mobile apps that use this information, which is called \emph{context}, are named context-aware applications.
For instance, an app that puts the phone in silent mode when the user is in a meeting is a context-aware app.
The user's location is a particularly useful context information because it allows apps to offer the user special functionalities when they are in a specific place.
The focus of this present work is location based applications.

Proximity-based is a particular type of context-aware that takes into account users' location.
These apps offer different possibilities, to the users, when they are in the proximity of a given point of interest.
We can find multiple examples of such apps.
Swarm\footnote{http://www.swarmapp.com} allows users to check-in in a given place or point of interest, according to user's location.
For instance, if the user is in a restaurant, he can use the app to perform a check-in in that restaurant and share his/her location on social networks.
These apps are becoming popular.
Besides these examples, it is possible to automate tasks based on location.
For instance, send a message to someone when we arrive at a given location or turn on the lights when we arrive at home.
Services such as \gls{IFTTT}\footnote{http://ifttt.com} allows users to make these kind of automations.
These are just examples of what is possible to do with context-aware and more specifically with location based applications.
With the right tools and applications we can benefit from our mobile devices being aware of context and location.

% Develop proximity-based
To develop proximity-based apps we need to, somehow, get the device's location.
However, we need to choose a technology to get this information and be able to associate data to points of interest, which we will refer as tags.
Besides location, we need a backend to store the data about each tag.
Developers have to develop the mobile apps, the backend and choose the right technology to get the location.

% Offer proximity-based services
Besides development of proximity-based services another question arrises.
How can an owner of a given place offer such services, without having to develop everything him/herself?
For instance, a store owner wants to advertise some promotion in the customers mobile devices when they approach the store.
How can he/she creates this promotion if there is no programming background?
Also, if there are multiple promotions in different stores, customers need to install one app for each store.
They should be able, by installing one app, to discover these promotions, or any other proximity-based service while they are walking, instead of being aware of these services and know which apps they need to install.

We created a tool to develop proximity-based services, removing the need to build a backend and to handle the location technology.
Here we introduce the concept of Smart Place.
The idea is to have a physical place, which offers a proximity-based service due to existence of tags placed on objects that are relevant.
For instance, the example of the store can be considered a Smart Place.
The owner place some tags, inside the store, that will trigger a notification in the customers' mobile devices, such as smartphones.

% Outline
Next section describes all the concepts needed in order to understand our Smart Places solution.
Section \ref{sec:related_work} presents related work about existing proximity-based applications and tools to develop such applications.
Then, we introduce our solution in section \ref{sec:solution}.
Section \ref{sec:evaluation} explains the evaluation of the solution being presented here.
Finnaly, section \ref{sec:conclusion} concludes this paper.
