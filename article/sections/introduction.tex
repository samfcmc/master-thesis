%!TEX root = ../article.tex

% Introduction
\section{Introduction}
\label{sec:introduction}
% Growing use of mobile devices
% Multiple sensors
% Apps can adapt according to context
The use of mobile devices have been increased in the last years.
It is possible to use devices, such as, smartphones and tablets, to perform tasks that, previously, were only possible using a computer.
These devices are becoming more similar to computers than ever before.
Not only they have powerful \glspl{CPU} and \glspl{GPU} but they are equipped with many sensors, such as, light sensor, \gls{GPS} and accelerometer.
Also, these devices have access to multiple data sources, such as,
the user's activity on social networks and user's calendar.
The applications (apps), that the user can install, have access to all this data, from the sensors and from the data sources.
Using this data, the apps can adjust settings and allow users to perform tasks according to it.
The mobile apps that use this information, which is called context, are named, context-aware applications.
There are multiple apps that use context, specially, the ones that use location.
Here, our focus will be on apps that depend on user's location.

Next, in section \ref{sec:introduction_motivation}, we introduce the motivation behind this thesis.
Then, in section \ref{sec:introduction_goals}, we establish some goals to meet with this thesis.
After, we state our contributions that aim to meet our goals, in section \ref{sec:introduction_contributions}.
Finnaly, in section \ref{sec:introduction_thesis_outline}, we outline this document's structure.

\subsection{Motivation}
\label{sec:introduction_motivation}
% Context-aware apps
% Proximity-based
There is a special kind of mobile app, which are, context-aware.
These apps get context information, from the device's sensors, such as the accelerometer.
It is also possible to get context information from data sources, such as, the user's calendar.
Using this information it is possible to build mobile apps that offer different possibilities according to context.
For instance, an app that puts the phone in silent mode when the user is in a meeting is a context-aware app.
Proximity-based is a particular type of context-aware that takes into account users' location.
These apps offer different possibilities, to the users, when they are in the proximity of a given point of interest.
We can find multiple examples of such apps.
Swarm\footnote{http://www.swarmapp.com} allows users to check-in in a given place or point of interest, according to user's location.
For instance, if the user is in a restaurant, he can use the app to perform a check-in in that restaurant and share his/her location on social networks.
Skout\footnote{http://www.skout.com}
is another example. It allows to find nearby users and start talking to them.

% Develop proximity-based
To develop proximity-based apps we need to, somehow, get the device's location.
However, we need to choose a technology to get this information and be able to associate data to points of interest, which we will refer, as tags.
Besides location, we need a backend to store the data about each tag.
Developers have to develop the mobile apps, the backend and choose the right technology to get the location.

% Offer proximity-based services
Besides development of proximity-based services, another question arrises.
How can a owner of a given place offer such services, without having to develop everything him/herself?
For instance, a store owner wants to advertise some promotion, in the customers mobile devices, when they approach the store.
How can he/she creates this promotion if there are no programming background?

% Use those services
Taking into account the previous store example, if there are multiple promotions in different stores, customers need to install one app for each store.
They should be able, by installing one app, to discover these promotions, or any other proximity-based service while they are walking, instead of being aware of these services and know which apps they need to install.

\subsection{Goals}
\label{sec:introduction_goals}
% Tool to develop proximity-based services without worrying about technologies...
As already mentioned, in order to create proximity-based services, available on mobile devices, developers need to build the mobile apps, choose the right technology for location and build the backend to maintain the data associated to each tag.
Our main goal, in this thesis, is to create a tool to develop such services, eliminating the need to build a backend and to choose the technology.
With this tool, developers should be able to build the proximity-based services, without the need to develop their own backends and handle the location technology themselves.

% Introduce the concept of Smart Place
In this thesis, we want to introduce the concept of Smart Place, which is described, in further detail, in the next chapter.
The idea is to have a place, which offers a proximity-based service due to existence of tags placed on objects that are relevant.
For instance, the example of the store, introduced in section \ref{sec:introduction_motivation}, can be considered a Smart Place.
The owner place some tags, inside the store, that will trigger a notification in the customers' mobile devices, such as, smartphones.

% Owners and users solution
Developers are not the only category of people we aim to target.
Owners of places, where they want to provide a proximity-based service, should have a way to decide which services they want to provide and customize those services.
For instance, in the example of the store, the owner should be able to get the service of advirtising promotions running and customize the promotions.
Here, we try to offer an interface, to owners, where they can manage which services they want to provide.
Also, users should be able to access these services without having to install, on their mobile devices, one app for each service they want to use.
In the store example, users should be able to have access to the advertised promotions without the need of one app for each store.
One app should be enough, for anybody with a mobile device, such as, a smartphone, to use any proximity-based service, that is, a Smart Place where owners placed tags with useful meaning.

\subsection{Contributions}
\label{sec:introduction_contributions}
% Easy way of developing proximity-based services
% Adapt existing applications (web)
% Easy way of installing such services
% Easy way of using them
% Concept of Smart Place
% App for owners
% App for users
In the beginning of this thesis, we aimed to contribute in several ways.
First, the tool to develop proximity-based applications, that we designed and built, can be considered as a contribution.
There are other tools that try to solve the same problem.
We analyzed them in order to understand their advantages and limitations and built our own solution, which tries to make developers focus on their applications and not on the technology side, that is, the technology used to provide tags and make mobile apps being able to detect them.
Our solution is based on the concept of Smart Place, which is, essentially, a place, where some tags are placed and different possibilities are offered, to the users, when they are on the proximity of those tags.
The introduction of this concept can be also considered a contribution of this thesis.
Around this concept, we have two kinds of people.
One one side, owners, who manage such places and are responsible to place tags, on the right place and to decide which proximity-based services they want to provide to people who visit their places.
On the other side, there are users, who are anyone, with a mobile device capable of detecting these tags.

Besides making easier the development of proximity-based applications, we have taken into consideration the people who manage Smart Places.
In our solution, there is an interface, provided by a mobile app, that allows owners to manage their Smart Place. Owners can register tags and choose, from a list of available Smart Places, which ones they want to configure.
For instance, in the store example, the owner would use this interface to choose to provide the promotions service and configure the promotion that would be advertised, when the customers are nearby each tag.

Another contribution we can consider is the interface between, the users and the tags placed in Smart Places.
We built a mobile app that can detect tags and offer different possibilities, according to the detected tag.
This app allows to access instead any Smart Place, instead of users having to install one app for each Smart Place.

\subsection{Thesis Outline}
\label{sec:introduction_thesis_outline}
% Outline all other chapters
After getting the motivaton and established some goals, we stated the contributions of this master thesis.
To be able to reach those contributions, a solution was implemented and evaluated.
First, we give some background to have the necessary information to understand the solution and the decisions behind it.
This document is organized as follows:
\begin{itemize}
  \item Chapter \ref{chapter:background}
  provides background information, that is needed to understand the problem, this thesis tries to solve and the solution;
  \item Chapter \ref{chapter:solution}
  introduces the solution and describes its main components;
  \item Chapter \ref{chapter:implementation}
  presents the implementation process, the solution's architecture and all the tools used to create it;
  \item Chapter \ref{chapter:evaluation}
  shows the results, from the evaluation process, in order to get a good insight about the quality of the solution;
  \item Chapter \ref{chapter:conclusion}
  concludes this thesis and states the limitations and what can be done, in the future, to improve this solution.
\end{itemize}
