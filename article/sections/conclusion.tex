%!TEX root = ../article.tex

% Conclusion
\section{Conclusion}
\label{sec:conclusion}
% Established some goals (which ones)
% Concept of Smart Place
% Types of location (taxonomy)
% Solution
% Implementation
% Evaluation
We have developed the Smart Places solution which offers a tool, for developers, to allow them to create proximity-based services and an interface for end users and owners.
We introduced the concept of Smart Place which is a pyshical space with tags that mobile devices can detect and offer different possibilites to the users, according to each tag.
We chose \gls{BLE} beacons using iBeacon protocol because it is compatible with any smartphone with Bluetooth version 4.0 or above.
The owners of Smart Places are responsible to place those beacons in the right place.
Users do not need any extra hardware.
They only need to download a single app and turn on the Bluetooth receivers of their mobile devices.

We have created a solution that allows developers, of Smart Places, that provide proximity-based services, using web technologies, such as, \gls{HTML}, \gls{CSS} and Javascript.
With this tool, any web application can react to the presence of tags placed in a given Smart Place.
This solution makes the development of Smart Places easier due to the fact that, developers do not need to handle the technology to handle tags neither the backend where the information about Smart Places and their tags is stored.
There is a mobile app for end users, anyone with a mobile device, capable of reacting to tags in a Smart Place.
Since each Smart Place is a web application they can run inside an embedded web browser in this mobile app, allowing users to have access to any Smart Place withouth the need to install a new native app for each one.
There is a mobile app for owners of Smart Places that they can use to choose which proximity-based services they want to provide and configure the necessary tags.

Two examples of Smart Places were created, using our solution, a Smart Restaurant and a Smart Museum.
The Smart Restaurant had the goal to allow customers, of a restaurant, to place their orders, after they take a sit using their mobile devices.
Using this solution, customers do not need to specify which table the orders belong to. Using the tag system through \gls{BLE} beacons, the app automatically get the table's number and add that information when the customer places the final order.
In the Smart Museum we created the experience of a museum where visitors can have access to more information about an object that they are close to.

We evaluated our solution in terms of battery consumption.
Users will not use our solution if they run out of battery faster than if they are not using our solution.
For each experiment, we have tried two different types of data connection, using \gls{WiFi} and \gls{3G}.
Using \gls{WiFi} the power drain is almost zero.
However, using \gls{3G}, the power drain was above 2\%, in just one hour.
Comparing with the power drain of Facebook app running in the same period we can conclude that the power drain is acceptable but there is room for improvement.
For instance, we could store geographical coordinates for each tag.
When detecting a tag, the app could use these coordinates to fetch data about the other tags in the same area in just one request.
This way, we need less requests to the backend which can result in less power drain than the actual solution.

\subsection{Future Work}
\label{sub:conclusion_future_work}

There are aspects of our work that can be improved in the future.
Owners of Smart Places need to deploy tags and use the mobile app to configure those tags.
However, there is not any interface, that they can use, to register themselves, as the owners of such tags.
In the development of this work we introduced the needed data manually, that is,
all the associations between tags and their owners.
There is one more interface missing to allow developers to register the Smart Places they develop.
When owners are configuring their Smart Places, they can choose from a list.
The Smart Restaurant and Smart Museum examples were introduced in the backend manually.
There is no way for developers to add a Smart Place to this list.
A future improvement could be these missing interfaces.
One that would allow owners to register the ownership of tags and another for owners that would use it to register the Smart Places they developed.
Another limitation is the fact that, using our solution, only web applications can offer proximity-based services.
In one side, this is a good fit because this is what allows to have one app to access multiple Smart Places.
On the other side, there could be developers that could use our solution to add proximity-based features in their existing native mobile apps.
This could be solved having an \gls{SDK} available for the three most used mobile \glspl{OS}, iOS, Android and Windows Phone, to allow to integrate our functionality in existing apps.

Proximity-based applications are an important type of context-aware applications since they provide services, to the user, that are relevant in the right place.
However, developers need the right tools to create these applications and users need an easy way to have access to such applications.
Our work targets not only developers but also users and owners.
A complete solution is needed because developers would not create proximity-based applications if there are no users.
Also, users would not be engaged if developers do not have the tools to create these applications.

Proximity-based applications can offer several possibilities.
For instance, companies that have online and physical stores could achieve more integration of offers such as, customers have access to exclusive discounts when they approach a given tag or a set of tags.
Another possibility is to have tags in a supermarket, which could trigger reminders in an app that customers use to store their shopping list.
For instance, the customer approaches the section where there is milk and this is on his shopping list. The app would remind him that he needs milk.
Our Smart Places solution and its evaluation can be considered a step forward to be able to use, develop and manage all these and many more possibilities.
