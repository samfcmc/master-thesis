%!TEX root = ../article.tex

% A new section
\section{Related Work}
\label{sec:related_work}
% Related work...
Here we discuss some related work in order to have a good insight, about the possibilities and existing solutions, for the problem we are trying to solve.
First, we take a look at some examples of proximity-based apps, in order to see what can be done when the user is in proximity of a given point of interest.
Since we tried to create a solution that provides an easy way of creating proximity-based services, there is a need to look at the state of the art of existing solutions of frameworks to develop context-aware applications. As already mentioned, proximity-based applications are a particular category of context-aware applications that take into account the user's location.

We present some
applications, where \gls{BLE} beacons were used, to get good insights about the potential use cases of this
technology:
\begin{itemize}
  \item BlueSentinel\cite{Conte2014}:
  This is a
  occupancy detection system, for smart buildings,
  that uses \gls{BLE} Beacons to detect the presence of
  people. The concept of a smart building
  is similar to Smart Place,
  due to the existence of sensors and actuators.
  It is focused on the power efficiency of the
  building. The idea is to optimize energy
  consumption according to people's presence.
  For instance, if there are no people in a given room,
  the heating system can be turned off.
  \item ContextCapture\cite{Antila2011}:
  In this work, the authors try to use
  context-based information to allow users to
  add more information to their status updates
  in the main social networks, such as
  Facebook\footnote{http://www.facebook.com} and
  Twitter\footnote{http://twitter.com}.
  The authors implemented a mobile app and a
  server integrated with Facebook and Twitter.
  Context information comes from the smartphone itself,
  from its sensors and from the nearby devices through
  Bluetooth.
  Devices can be other smartphones or \gls{BLE} Beacons, which
  are used for indoor location.
\end{itemize}

Since we have created a framework to develop
proximity-based services, the state
of the art of existing frameworks, that deliver
context information to the apps will be presented:
\begin{itemize}
  \item Frameworks for developing distributed
  location-based applications:
  There are frameworks to develop location-based
  applications.
  In the work presented in Krevl et al.\cite{Krevl2006},
  a framework
  was developed to allow developers to build
  location-based apps. Location information can come
  from any source, such as \gls{GPS} receivers, Bluetooth
  receivers and \gls{WiFi} receivers.
  The authors discuss some benefits and limitations
  of several technologies for getting the
  user's positioning.
  The mobile device get geographical coordinates
  from any source and send that data, in a
  \gls{SOAP}\cite{Seely:2001:SCP:560836} message,
  to the appropriate web service in the Application
  Server. This server, communicates with the Database Server
  to query the database, which sends back a response with
  location information, if there is any, for that
  particular group of geographical coordinates.
  The authors did not evaluate the system.
  This solution offers abstractions for the location
  information sources. Geographical coordinates can
  come from any source.
  The authors do not take into consideration
  constraints in terms of resources, such as
  lack of Internet connection and battery.
  Since most users have limited data plans for
  their smartphones and \gls{SOAP} messages can
  grow, in size, due to its \gls{XML} format,
  a more efficient message encoding could be used
  instead.
  \item Dynamix\cite{Carlson2012}:
  is a framework to develop
  mobile native and web apps that allow them to receive
  context information, for instance, position and device's
  orientation. This framework has plugins that get
  one or more sensor's raw data and turn that into event
  objects, that contain more high-level information.
  This framework supports many kinds of context information
  and it is possible to develop more plugins to allow the
  apps to generate additional events that are not
  already supported.

  To achieve our goal, our framework could be just a
  plugin for Dynamix. The plugin would
  need to get the beacon's raw data and
  turn that into a more high-level information
  using a backend. In this framework,
  the user needs to install an app, that manages the service
  that runs in background, and needs to define some
  security policies. This could mean a big overhead since
  we are more focused on developing proximity-based applications, that do not require such complex security
  policies because, in this kind of applications, there is only need
  to access the device's sensors that could provide,
  to the applications, positioning data.
\end{itemize}
