%!TEX root = ../article.tex

% Abstract
\begin{abstract}
  Proximity-based applications engage users while they are in the proximity of points of interest. These apps are becoming popular among the users of mobile devices.
  Proximity based apps are triggered when the user is at specific geographic coordinates, but more interestingly, they can be triggered by tagged objects.
  We introduce the concept of Smart Place, which is a physical place, where tags are placed, in order to provide a service. Users, with a mobile device capable of detecting such tags can use the provided service when they are in proximity of these tags.
  Several technologies can be used to create tags.

  In this dissertation, we analyze several location technologies in order to choose one that best fits the concept of Smart Place.
  We have created a solution to develop proximity-based applications, based on this concept.
  Also, two mobile apps were developed.
  Two examples of Smart Places were built, the Smart Restaurant and Smart Museum.
  The app for users allows anyone, with a mobile device, capable of detecting tags, to have access to any proximity-based service, based on our concept of Smart Place and created using the tool we developed.
  This app was evaluated in terms of energy consumption and the results show that the battery drain can be acceptable, in a daily basis when the user is using WiFi. However, using 3G the battery drains more than 200 times faster.
\end{abstract}

% Keywords
\begin{keywords}
  proximity-based; mobile apps; smart place; location based applications
\end{keywords}
