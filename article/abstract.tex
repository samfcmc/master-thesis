%!TEX root = ../article.tex

% Abstract
\begin{abstract}
Proximity-based applications engage users while they are in proximity of points of interest. These apps are becoming popular between the users of mobile devices.
However, instead of just having fixed points, it is possible to use a more generic approach, where a proximity-based application can use tags. These tags can belong to objects that are in a fixed position or even to moving objects.
Here, we introduce the concept of Smart Place, which is a place, where tags are placed, in order to provide a service. Users, with a mobile device capable of detecting such tags, can use the provided service, when they are in proximity of these tags.
Several technologies can be used to create tags.
In this thesis, we analyze several location technologies in order to choose one that best fits the concept of Smart Place.
We have created a solution to develop proximity-based applications, based on this concept.
Also, two mobile apps were developed.
One for owners of Smart Places and other for users.
The first one allows owners, after installing tags, to choose which proximity-based services they want to provide and configure tags according to the provided services.
The app for users allows anyone, with a mobile device, capable of detecting tags, to have access to any proximity-based service, based on our concept of Smart Place and created using the tool we developed.
This app was evaluated in terms of energy consumption.
Two examples of Smart Places were built.
\end{abstract}

% Keywords
\begin{keywords}
  proximity-based; mobile apps; smart place; location
\end{keywords}
