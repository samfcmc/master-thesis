%TEX root = ../dissertation.tex

\chapter{Background}
\label{chapter:background}
% Context-aware applications
% proximity-based apps
% Technologies
Before describing the solution, we need to get a good insight about concepts, such as, context-aware applications, technologies available and related work.
We built a framework to develop proximity-based applications.
However, we need to define the concept of proximity-based application.
These are a particular kind of context-aware applications.
An application is context-aware when it takes into account the context, such as, location, device's orientation, temperature, etc.
Based on this definition, proximity-based applications are context-aware applications that take into account the user's location.
They engage the user while they are on proximity of a given point of interest, which we will call a Tag.
Someone installs tags in a given space.
Then, users can interact with those tags when they are nearby them.

From the given definition of proximity-based application, the concept of Smart Place arrises.
However, since this work is based on location, we need to look at a taxonomy, described in section \ref{sec:background_location}, that allows us to classify the multiple kinds of location to justify our decisions in terms of solution and implementation.
Our solution is based on the concept of Smart Place, which we define, in further detail, below, in section \ref{sec:background_smart_places}.
Then, we explore some technologies that could be used in our solution, in section \ref{sec:background_technologies}.
Finnaly, we present related work,
in section \ref{sec:background_related_work} that is, some existing solutions for the same problem we try to solve in this thesis.

\section{Location}
\label{sec:background_location}
% Techniques
% Physical vs symbolic
% Absolute vs Relative
% Computed
% Scale
% Recognition
% Cost
% Limitations
In order to understand what a Smart Place is, we need to have a good insight about concepts related to location.
We have used a taxonomy\cite{location} to classify some properties about location.
These properties will allow us to have a better understanding about the concept of Smart Place, in section \ref{sec:background_smart_places}.
Also, these properties are needed to be able to classify the technologies, described in section \ref{sec:background_technologies} and justify if they can be used to implement the concept of Smart Place.
Using the taxonomy previously mentioned, it is possible to classify location systems in terms of techniques, physical position or symbolic location, absoulute or relative position, location computation, scale, recognition, cost and limitations.

\subsection{Techniques}
\label{sub:background_techniques}
There are three techniques that can be used to get the device's location.
These techniques are used to compute the location.
A device can use one or combine two or all the three.
Using one technique, does not mean that only that one can be used.
The techniques are the following:
\begin{description}
  \item[Triangulation] It can be done via lateration or angulation. Lateration uses distance measurements between two well known points.
  Angulation uses measurements of angles relative to known points;
  \item[Proximity] As the name suggests, it measures how close the device is to a known point;
  \item[Scene analysis] It consists in examining a view from a given point
\end{description}

As already mentioned, these techniques can be combined.
For instance, \gls{GPS}, described in section \ref{sub:background_gps}, uses this technique.
Systems based on tags, such as \gls{NFC}, explained in section \ref{sub:background_near_field _communication}, are based on Proximity.

\subsection{Physical or symbolic position}
\label{sub:background_physical_or_symbolic_position}
The device's position can be classified in one of the two types:
\begin{description}
  \item[Physical position] This kind of position refers to a given set of coordinates that, identifies, unequivocally, a place, usually on Earth, where the device is. From this set of coordinates, we know, exactly, where the object is;
  \item[Symbolic location] Unlike physical location, using this kind of positioning, it is not possible to identify where the object is. Symbolic can be, for instance, the object is in the kitchen, office, etc. The meaning of its position depends on the application.
\end{description}

One location system can only be classified in one of the two kinds of positioning.
However, physical position can be augmented to also have symbolic information.
For instance, we can have a system that stores \gls{GPS} coordinates and, for each set of coordinates, we store symbolic information.
A place on Earth could have symbolic information associated.

\subsection{Computation}
\label{sub:background_computation}
We have multiple ways of obtaining data, about location and multiple kinds of location.
However, this data needs to be computed, somehow, in order to get meaning information about the object we want to locate.
This computation can be done in one of the two ways:
\begin{description}
  \item[Located object computes its own location] Here, the located object has means to compute its location itself;
  \item[Location is computed by external infrastructure] The located object delegates its location computation to an external infrastructure, for instance, a central server.
\end{description}

For instance, \gls{GPS} receivers compute their own location based on signals that come from satellites.
In location systems based on tags, the location's computation is delegated to another, usually, more powerful machine.

\subsection{Scale}
\label{sub:background_scale}
The scale of a location system refers to the number of objects that is possible to locate using a certain amount of infrastructure or over a given period of time.
For instance, in systems that rely on sattelites, such as the \gls{GPS}, with a fixed number of sattelites, it is possible to serve an unlimited number of receivers.
In systems based on tags, a reader can read a limited number of tags.
In this case, adding more tags can compromise the performance of the entire system.

\subsection{Recognition}
\label{sub:background_recognition}
Recognition refers to the hability, of a location system, to recognize individual receivers.
Location systems based on tags, usually, have means of recognizing receivers.
If we assign an \gls{ID} to each receiver, each time the receiver reads a tag, that information can be stored and it is possible to know which receiver communicated with that tag.
There are systems without this capability.
For instance, \gls{GPS} is a system that is not able to recognize receivers.
The sattelites do not have any means to recognize receivers.

\subsection{Cost}
\label{sub:background_cost}
There are costs associated to any location system.
It is possible to look at costs in three perspectives:
\begin{description}
  \item[Time costs] Any location system needs time to be spent in installation, administration and other tasks related with its maintenance;
  \item[Space costs] Its always some infrastructure associated with any location system. This infrastructure needs space. For instance, if servers are needed, we need to install them in some room;
  \item[Capital costs] The processes and infrastructure associated to a location system requires capital.
  Each mobile unit or infrastructure element has its price. Also, sometimes, there are people involved. Their salaries are also a capital cost that needs to be taken into consideration.
\end{description}

When comparing multiple location systems, the costs can be a decisive factor. We need, not only to take into account the technology characteristics but also the costs.

\subsection{Limitations}
\label{sub:background_limitations}
There are no perfect location systems. Every system has its own limitations.
There are ones that do not work well indoors, such as, the \gls{GPS}, because it cannot get the sattelites' signal.
If we want a location system that would work indoor, maybe a tag based system is a better fit.
When choosing a system, this is important because, one might not require much hardware and it leads to a low cost system but, if the limitations compromise its performance, it is better to pick one with higher costs.
For instance, a tag based system might only work properly if only one tag is present.

\section{Smart Places}
\label{sec:background_smart_places}
% Popularity of proximity-based...
% Based on proximity-based
% Explain owners and users
% Figure
% Owner puts tags...
% Relate with location characteristics
As already mentioned, proximity-based applications engage users while they are on proximity of some point of interest.
Our solution is based on the concept of Smart Place.
A Smart Place is some place, with tags, which users can interact with, using a mobile device, such as, a smartphone.
Place's owner installs tags and those tags offer some service to the users when they are nearby.
For instance, a store owner could use these tags to advertise some promotion when the customers are nearby the store.
However, Smart Places are not just about stores.
This concept can be used to any kind of service that requires the users to be nearby tags.
For instance, it is possible to build a Smart Restaurant, where tags have the information about the table's number and the customers are allowed to call a waiter without requiring to type the table's number.

Multiple kinds of people are involved in Smart Places.
There are owners, developers and end users.
Figure~\ref{fig:smart_places_overview} shows how these three kinds of users interact with a Smart Place.
Owners are responsible for managing and installing tags in their places, where they want to offer a proximity-based service.
Also, developers are needed to develop these services, for instance, the Smart Restaurant or the store promotions.
Finally, the end users that interact with tags, that are installed in Smart Places, using their mobile devices.

\begin{figure}[!ht]
  \centering
    \includegraphics[width=0.8\textwidth, keepaspectratio]{images/smart_places_overview}
    \caption{Smart Places Overview}
    \label{fig:smart_places_overview}
\end{figure}

In terms of classification, since Smart Places are about proximity, it is reasonable to pick proximity as the technique and symbolic location as the type of location.
These are the main characteristics we need to take into consideration in order to choose one technology to support this concept.
In section \ref{sec:background_technologies} it is provided an overview of existing technologies and which ones can be used based on the fact that we want to provide symbolic location.
Also, cost and limitations need to be taken into account because we do not want a solution that implies a high cost for the user.
We want the low cost as possible, in terms of time and capital, for owners and developers.

\section{Technologies}
\label{sec:background_technologies}
% Possible technologies
% GPS
% QR Codes
% NFC
% Google maps indoor
% BLE
As previously mentioned, a Smart Place has tags and the users can interact with them using their mobile devices.
Somehow, the mobile device needs to be able to detect the presence of these tags.
Multiple technologies can be used.
Some of them require the user interaction, such as the described in section \ref{sub:background_qr_codes}.
Others require the devices to be equipped with extra hardware, such as the one in section \ref{sub:background_near_field _communication}.
We are going to look at these technologies and see which one best fits our purpose, using the taxonomy introduced in section \ref{sec:background_location}.

\subsection{\glstext{GPS}}
\label{sub:background_gps}
% What is it
% How it works
\gls{GPS} is a location system that uses 24 sattelites plus three backups.
Receivers send signals and sattelites answer back. Measurements are taken from this measurements in order to receivers be able to calculate their own location.
An object, that we want to locate, only needs a \gls{GPS} receiver in order to be located using this technology.

% Classification
Using the taxonomy, explained in section \ref{sec:background_location}, \gls{GPS} uses triangulation as the technique to get the location.
It computes physical location, that is, when an object is located, we can look at a map and see where it is.
The located objects have means to compute their own location, based on measurements from the sattelites' signals.
It is a scalable system because, the same number of sattelites can handle an unlimited number of \gls{GPS} receivers.
The sattitles are not able to recognize individual receivers.
In this system, the major cost is in maintaining the sattelites and, in the beginning, was launching the sattelites and, in the future, it might be necessary to replace the existing ones.
It cost 12 billion dollars to put them in orbit. The annual operating cost is 750 million dollars\footnote{Source: http://nation.time.com/2012/05/21/how-much-does-gps-cost at 24, December, 2015}.
One limitation of \gls{GPS} is that, it does not well indoors because the sattelite's signal can be really weak.

% Requirements
% Advantages and disadvantages
Objects to be located using \gls{GPS} only require an adequate receiver. Most of the mobile devices, nowadays, are already equipped with these receivers.
This can be a really big advantage, because, it is a low cost solution for users.
However, as already mentioned, it is not adequate to locate objects indoors.

% Usage for Smart Places
As already mentioned, the concept of Smart Place is based on proximity and symbolic location.
Also, it is supposed to work indoors.
\gls{GPS} is a location system that uses triangulation to provide physical location.
It is possible to augment physical location in order to provide symbolic.
However, it only works properly where it has a good receiption of the sattelites' signal.
Since the concept of Smart Place is a concept connected to an indoor space, the \gls{GPS} is not a good option, even if we augment it to provide symbolic location, because it does not work properly indoors.

\subsection{QR Codes}
\label{sub:background_qr_codes}
% What is it
% How it works
\gls{QR} codes is a type of two dimensional barcode.
The user just needs an app that reads these codes.
There are \gls{QR} readers available for all the three major mobile platforms, Android, iOS and Windows Phone.
Whenever the user sees one of these codes, he/she opens the \gls{QR} code reader app, scan the code and see the content provided by it.
The content can be an \gls{URL}.
Figure~\ref{fig:qr_code} shows an example of a \gls{QR} code.

\begin{figure}[!ht]
  \centering
    \includegraphics[width=0.3\textwidth, keepaspectratio]{images/qr_code}
    \caption{Example of a QR Code}
    \label{fig:qr_code}
\end{figure}

% Requirements
% Advantages and disadvantages
It does not require more than a mobile device with a camera and a \gls{QR} code reader app, such as, QR Droid\footnote{http://play.google.com/store/apps/details?id=la.droid.qr} for Android, QR Reader\footnote{http://itunes.apple.com/pt/app/qr-reader-for-iphone/id368494609?mt=8} for iOS and QR Code Reader\footnote{http://www.microsoft.com/en-us/store/apps/qr-code-reader/9wzdncrfj1s9} for Windows Phone.
There is no extra hardware involved.
However, one big disadvantage of this technology is the fact that it requires the user interaction.
The user needs to be aware of this kind of codes and also needs to see them wherever they are.
Providing proximity-based services using \gls{QR} codes would be simply using the codes as tags and an app for users to scan these codes and have access to the particular service, that is, Smart Place.

% Related work
There are works that aim to use this technology as a mean to get the user's indoor location, such as, the one described in \cite{qr_indoor}.
It uses \gls{QR} codes in combination with \tm{Google} Maps \gls{API}\footnote{http://developers.google.com/maps}.
Others, try to use it to solve real world problems, such as, hospital overcrowding.
In the work, presented in \cite{qr_hospital}, the authors try to use \gls{QR} Codes to identify patients, in an hospital.
This information is used in a mobile application that the hospital's staff use to register activities related to the patient.

% Classification
If we look at \gls{QR} codes as tags, this can be considered as a location system.
In terms of techniques, it uses proximity, because, only when the user approaches and scans the code the \gls{QR} code can be considered as a tag being read.
It can provide symbolic location if information about each code is stored.
In order to provide this kind of location, the computation needs to be performed in external infrastructure.
In terms of scale, it depends on the characteristics of the infrastructure, such as, the hardware specifications of the servers where the location is computed.
It is possible to recognize individual receivers if we store enough information for that.
The costs depend on the amount of external infrastructure. There are servers that need to be installed and maintained.
Also, the app that handles the code scanning needs to be developed and maintained. The codes can be shown in just a piece of paper.

This technology could be used in our solution.
However, as already mentioned, it requires the users to be aware of such codes and start the scanning process themselves instead of being an automatic process happening in background.

\subsection{NFC}
\label{sub:background_near_field _communication}
% What is it
% How it works
\gls{NFC} is a short distance radio communication technology.
The two devices communicating need to be of 10 cm or less distant from each other.
Each \gls{NFC} device can work in one of three modes:
\begin{description}
  \item[Card Emulation] In this mode, devices act as they were smart cards, allowing to perform transactions, such as payments;
  \item[Reader/Writer] Devices are enabled to read information from \gls{NFC} tags, embedded in other objects, such as, posters;
  \item[Peer-to-peer] Devices exchange information with each order in an adhoc way.
\end{description}

% Requirements
% Advantages and disadvantages
Simillary to \gls{QR} Codes, described in section \ref{sub:background_qr_codes}, this technology requires the user interaction and his/her awareness of the existence of these kinds of tags.
Also, devices need to be equipped with \gls{NFC} readers.
% Related work
This technology is already being used for payments, such as \tm{Apple} Pay\footnote{http://www.apple.com/apple-pay/}.

% Classification
\gls{NFC} is based on the proximity technique and it can provide symbolic location if enough information is stored in external infrastructure.
The location computation is delegated to that infrastructure.
Since it is a tag based system, its scale is dictated by the infrastructure where the most computation is done and not by the device that is being located.
Individual devices can be recognized if their are identified and that information is stored in the external infrastructure.
The costs depend on the amount of infrastructure. Also, there is a need to install \gls{NFC} tags to be read by the readers installed on mobile devices.
The main limitation is that it is only possible to read a tag at a time. It is not possible to read multiple tags in one reading.

% Usage in Smart Places
It could be used for Smart Places.
It is based on proximity and it can provide symbolic location, which are the foundations of Smart Places.
However, this would lead to a solution that would require the user interaction to discover proximity-based services.
The user would need to be aware of \gls{NFC} tags.

% Get references to this...
\subsection{Google Maps Indoor}
\label{sub:background_google_maps_indoor}
% What is it
% How it works
Google Maps is a service, from \tm{Google}, to see maps.
It allows users to navigate all over the world and even gives access to satellite images.
There are mobile apps, for all most used mobile \glspl{OS}, Android, iOS and Windows Phone.
With more than 1 000 000 000 installs on Google Play Store and an average rating of 4.3\footnote{Source: http://www.appannie.com/apps/google-play/app/com.google.android.apps.maps in 23 December 2015}, we can say it is a very popular and mainstream app.
When installed on a smartphone, it uses multiple sensors, such as, \gls{GPS} and accelerometer, in order to get the user's location.
However, since it relies on \gls{GPS}, it does not work well indoors.
Google Maps has an extension wich is, Google Maps Indoor\footnote{http://www.google.com/maps/about/partners/indoormaps}, which allows the user to navigate inside a building.
Figure~\ref{fig:google_maps_indoor} shows an example of Google Maps Indoor.

\begin{figure}[!ht]
  \centering
    \includegraphics[width=0.3\textwidth, keepaspectratio]{images/screenshots/google_maps_indoor}
    \caption{Screenshot of Google Maps with indoor functionality}
    \label{fig:google_maps_indoor}
\end{figure}

% Requirements
% Advantages and disadvantages
This service uses sensors already provided by the mobile device and it does not require extra sensors inside the building.
However, buildings' owners need to provide the necessary data to \tm{Google}.
Not all buildings are available.

% Classification
Similary to \gls{GPS}, described in section \ref{sub:background_gps} Google Maps Indoor is based on triangulation technique.
It provides physical location, because it is possible to locate the device inside a building.
To compute the location, it uses resources available on the device itself in combination with servers.
Its scale depends on the external infrastructure, such as the servers where the computation is performed.
It is possible to recognize individual devices.
In terms of costs, there are no much costs for the user. The necessary infrastructure is the major cost here.
The main limitation is that, it requires the building to be mapped by \tm{Google}. A team needs to go the building and get the necessary information to make it available.

% Usage in Smart Places
Smart Places could be implemented using this technology.
This would lead to a solution where no extra hardware is needed.
Google Maps Indoor provides physical location. However, as any physical location system, it can be augmented to provide symbolic location.
Unfortunately, as already mentioned, not all buildings are available.

\subsection{Bluetooth Low Energy}
\label{sub:background_bluetooth_low_energy}
% What is it
% How it works
\gls{BLE}\cite{ble} is a short range wireless communication technology, developed by Bluetooth \gls{SIG}\footnote{http://www.bluetooth.org}.
Unlike classic Bluetooth, it is focused on low power consumption.
It is a feature of Bluetooth 4.0\cite{bluetooth_specification}.
To take advantage of this technology, the mobile device needs to be equipped with Bluetooth, at least, version 4.0.
However, to be able to use it, in order to get the user's indoor location, a protocol is needed.
There are two protocols that can be used: iBeacon\footnote{http://developer.apple.com/ibeacon}, developed by \tm{Apple} and Eddystone\footnote{http://github.com/google/eddystone}, developed by \tm{Google}.

% Protocols: ibeacon and eddystone
The iBeacon protocol works with small devices, named beacons, that broadcast a sequence of bytes, which acts as an identifier allowing to build proximity-based applications\cite{ibeacon_book}.
Figure~\ref{fig:ibeacon_message} shows the structure of this sequence of bytes, where it is possible to see three parts:
\begin{description}
  \item[\gls{UUID}] has 16 bytes (128 bits) and it identifies the organization that the beacon belongs to;
  \item[Major] has two bytes (16 bits) and it identifies a group of beacons that belong to a given organization identified by the \gls{UUID};
  \item[Minor] has two bytes (16 bits) and it identifies each individual beacon in the group identified bu the Major value.
\end{description}
In this protocol, location is reported to an application using one of the two operations:
\begin{description}
  \item[Monitoring] This operation is called when the beacon and the mobile device are in the same space;
  \item[Ranging] It is related to a single beacon. The distance from a mobile device to a beacon is estimated using its transmissions.
\end{description}

\begin{figure}[!ht]
  \centering
    \includegraphics[width=0.6\textwidth, keepaspectratio]{images/ibeacon_message}
    \caption{Structure of the sequence of bytes transmitted in iBeacon protocol}
    \label{fig:ibeacon_message}
\end{figure}

In the other mentioned protocol, Eddystone, simillary to iBeacon, the beacons also advertise a sequence of bytes.
However, there are three types of messages, that beacons can broadcast to mobile devices nearby\footnote{http://github.com/google/eddystone/blob/master/protocol-specification.md}:
\begin{description}
  \item[Eddystone-UID] It broadcasts an unique 16 byte (128 bits) identifier. The first 10 bytes are for the namespace, which is used to distinguish groups of beacons, and the remaining 6 are for the instance \gls{ID}, which is used to identify individual beacons, inside the same namespace;
  \item[Eddystone-URL] As the name suggests, it broadcasts an \gls{URL};
  \item[Eddystone-TLM] Here, telematry information about the beacon is transmitted, such as, battery voltage and device's temperature.
\end{description}

% Requirements
% Advantages and disadvantages
The main advantage of this technology is that it only requires that the mobile device has Bluetooth, at least, version 4.0.
However, to develop proximity-based applications, we need to deploy some beacons that use the iBeacon or the Eddystone protocol.
Fortunately, the extra hardware is the space owner's responsability.
The user does not need anything else besides his/her mobile device.
Using this technology, if developers want to map beacons to more kinds of information, only the mentioned protocols are not enough.
We need a backend to store the mapping between the beacons and that information.
Besides developing the application itself, developers also need to deal with the backend and all the concerns around any distributed system, such as, scalability, performance, etc.

% Classification
Since the device only can detect beacons when it is on proximity, this technology is based on proximity technique.
It can provide symbolic location using some external infrastructure.
The device's location can be computed by that infrastructure.
Its scale depends on hardware specifications of the servers where the needed information is stored.
Also, using this information it is possible to recognize individual devices.
Besides the cost of developing the mobile application and the needed infrastructure, there also costs for owners of places, because they need to deploy beacons in order to allow users' mobile devices to detect them.
The main limitation is that, this solution requires the deployment of \gls{BLE} beacons.

% Usage in Smart Places
This technology is a good fit for our purpose because, it is based on proximity and can be used to provide symbolic location.
It does not have any costs for the users besides the acquisition of the mobile device.
We can place the beacons wherever we want and where it makes sense for the service we want to provide.
Also, using this technology, the mobile app that will handle the nearby beacons, can be running in background.
This way, the user does not to be aware of any tags and he/she can be notified if a proximity-based service is found.

\subsection{Outline}
\label{sub:background_outline}
% Main advantages of each one
% The chosen one (BLE ibeacon)
% Why
Multiple technologies were taken into consideration.
Ones are tag based, such as \gls{QR} codes, \gls{NFC} and \gls{BLE} that can be used to provide symbolic location.
Other ones, such as \gls{GPS} and Google Maps Indoor, are used to provide physical location but can be augmented to provide symbolic location.
However, \gls{GPS} does not work properly indoors and Google Maps Indoor requires that buildings are mapped first by a team from \tm{Google}.
Each technology was classified in terms of techniques, type of location (physical or symbolic), computation, scale, recognition of individual receivers, costs and limitations.
This classification was made using the taxonomy introduced in section \ref{sec:background_location} and is summarized in Table~\ref{tab:technologies}.

% Please add the following required packages to your document preamble:
% \usepackage{booktabs}
% \usepackage{graphicx}
\begin{table}[]
\centering
\resizebox{\textwidth}{!}{%
\begin{tabular}{@{}lllllll@{}}
\toprule
\textbf{Technology}                                                       & \textbf{Techniques} & \textbf{Type of location} & \textbf{Computation}                                                              & \textbf{Scale}                                                                        & \textbf{Costs}                                                               \\ \midrule
\textbf{GPS}                                                              & Triangulation       & Physical                  & In device                                                                         & \begin{tabular}[c]{@{}l@{}}24 satellites \\ serve unlimited \\ receivers\end{tabular} & Satellites                                                                   \\
\textbf{\begin{tabular}[c]{@{}l@{}}QR \\ Codes\end{tabular}}              & Proximity           & Symbolic                  & \begin{tabular}[c]{@{}l@{}}External \\ infrastructure\end{tabular}                & \begin{tabular}[c]{@{}l@{}}Depends on \\ the external \\ infrastructure\end{tabular}  & \begin{tabular}[c]{@{}l@{}}External\\ infrastructure\end{tabular}            \\
\textbf{NFC}                                                              & Proximity           & Symbolic                  & \begin{tabular}[c]{@{}l@{}}External \\ infrastructure\end{tabular}                & \begin{tabular}[c]{@{}l@{}}Depends on \\ the external \\ infrastructure\end{tabular}  & \begin{tabular}[c]{@{}l@{}}External\\ infrastructure\\ +\\ Tags\end{tabular} \\
\textbf{\begin{tabular}[c]{@{}l@{}}Google \\ Maps \\ Indoor\end{tabular}} & Triangulation       & Physical                  & \begin{tabular}[c]{@{}l@{}}In device\\ +\\ External\\ infrastructure\end{tabular} & \begin{tabular}[c]{@{}l@{}}Depends on\\ the external \\ infrastructure\end{tabular}   & \begin{tabular}[c]{@{}l@{}}External\\ infrastructure\end{tabular}            \\
\textbf{BLE}                                                              & Proximity           & Symbolic                  & \begin{tabular}[c]{@{}l@{}}External \\ infrastructure\end{tabular}                & \begin{tabular}[c]{@{}l@{}}Depends on \\ the external \\ infrastructure\end{tabular}  & \begin{tabular}[c]{@{}l@{}}External\\ infrastructure\\ +\\ Tags\end{tabular} \\ \bottomrule
\end{tabular}
}
\caption[Comparison of location technologies]{Comparison of location technologies}
\label{tab:technologies}
\end{table}


\gls{GPS} and Google Maps Indoor do not require extra hardware. However, they were created to provide physical location and not symbolic as it is needed for the concept of Smart Place.
\gls{GPS} does not work properly indoors and not all buildings are mapped in Google Maps Indoor.
\gls{QR} codes can be used for symbolic location. They do not need extra hardware and there tools to generate these codes\footnote{http://goqr.me}.
However, they require the user interaction. He/she needs to be aware of the existence of these codes and use his/her mobile device to scan the code and get access to the content it provides.

The final decision was to pick \gls{BLE} because it has the best tradeoff between cost and kind of location it provides. Using this technology we have symbolic location. With the adequated infrastructure it is possible to associate any kind of information to each tag.
It requires extra hardware but this is the responsability of who manages the place where the tags will be deployed.
The user does not anything else but a mobile device with Bluetooth, version 4.0 or later.
Three beacons, from \tm{Estimote} were used in the implementation of our solution, described in chapter \ref{chapter:implementation}.

\section{Related Work}
\label{sec:background_related_work}
% Related work...
Here we discuss some related work in order to have a good insight, about the possibilities and existing solutions, for the problem we are trying to solve.
First, we take a look at some examples of proximity-based apps, in order to see what can be done when the user is in proximity of some \gls{POI}.
Since we tried to create a solution that provided an easy way of creating proximity-based services, there is a need to look at the state of the art of existing solutions of frameworks to develop context-aware applications. As already mentioned, proximity-based applications are a particular category of context-aware applications that take into account the user's location.

\subsection{Proximity-based apps}
\label{sub:background_ble_beacons_applications}
% Why
% For each one: What, Pros, Cons
\gls{BLE} Beacons were
be used to develop our solution, based on the concept of Smart Places.
Some
applications, where this technology is used,
are be presented here, to
get good insights about the potential use cases of this
technology and the apps developed using it.
\begin{description}
  \item[BlueSentinel\cite{Conte2014}]
  This is a
  occupancy detection system, for smart buildings,
  that uses \gls{BLE} Beacons to detect the presence of
  people. The concept of a smart building
  is similar to Smart Place,
  due to the existence of sensors and actuators.
  It is focused on the power efficiency of the
  building. The idea is to optimize energy
  consumption according to people's presence.
  For instance, if there are no people in a given room,
  the heating system can be turned off.
  In this solution, the users have to install
  an app, that will get the beacons' signal and
  send data to a server, which will process it
  and send requests to actuators in order to
  perform actions to optimize the
  building's power efficiency.
  Unfortunately, there is a limitation
  of iBeacon protocol implementation
  in iOS.
  Beacons can be received, by the apps,
  only when these are active. When the apps are in
  background, they are waken up only to handle
  enter/exit region events. To circumvent this
  limitation, the authors developed custom
  beacons, which advertise more than one region
  in a cyclic sequence. These custom beacons
  were created using an
  Arduino\footnote{http://www.arduino.cc/}
  and an Bluetooth \gls{USB} dongle.
  Since this solution is a native app,
  users have to install it in order
  to make the smart building work to
  optimize power efficiency.
  Once the user starts the app, he/she does not
  need to interact with it anymore, since it
  will run in background.
  \item[BlueView\cite{Chen2013}]
  This one is a system to help
  visually impaired people to perceive some \glspl{POI}.
  This solution has two main components: The viewer device
  and the \glspl{BP}. The first one is a mobile phone,
  carried by the user, which is bluetooth enabled.
  The \glspl{BP} are just bluetooth tags instead of
  \gls{BLE} Beacons. The name of a \gls{POI} is associated with
  \gls{MAC} address of the tag which it is associated to.
  The steps involved in using the system are the
  following: first, the viewer device will scan
  for nearby \glspl{BP}; then, a list of the names of
  \glspl{BP} is created. This list is refreshed anytime a new
  \gls{BP} is detected and the user is informed through auditory
  feedback. The second step consists of the user, using
  the viewer's device, establishing a connection with a \gls{BP}
  attached to an object. Finally, using audio prompt, the \gls{BP}
  will assist the user in locating the object.
  Despite of this solution being a mobile app, installed
  in the viewer's device, the authors do not have in
  consideration the typical concerns of any mobile app,
  such as the energy consumption.
  The authors tested the application, in 2013,
  using Nokia N70 as the viewer device.
  This solution could be implemented using \gls{BLE} Beacons
  and the viewer device could be any Android, iOS or
  Windows Phone smartphone.
  \item[ContextCapture\cite{Antila2011}]
  In this work, the authors try to use
  context-based information to allow users to
  add more information to their status updates
  in the main social networks, such as
  Facebook\footnote{http://www.facebook.com} and
  Twitter\footnote{http://twitter.com}.
  This work had two main goals: first, demonstrate technical
  aspects of collaborative context, such as,
  how to get contextual information from
  surrounding devices and how they can be used
  as a source of contextual information;
  second, test and analyze the user experience of
  context-aware systems.
  The user can decide the abstraction level (coordinates,
  address or semantic label).
  The authors implemented a mobile app and a
  server integrated with Facebook and Twitter.
  Context information comes from the smartphone itself,
  from its sensors and from the nearby devices through
  Bluetooth.
  Devices can be other smartphones or \gls{BLE} Beacons, which
  are used for indoor location.
  Similarly to \cite{BenAbdesslem2014}, devices communicate
  with each other as a network.
  Using this solution, the user can create status updates,
  in the mentioned social networks, in the format shown Listing~\ref{lst:context_capture_status_update}:
  \begin{listing}[H]
      [User-defined message]

      Sent from [Location] while [Activity]

      [Description] [Topic] and [Applications Activity] with
      [Friends].
    \caption{Format of status updates in ContextCapture}
    \label{lst:context_capture_status_update}
  \end{listing}
\end{description}

\subsection{Context-awareness}
\label{sub:frameworks_context_aware}
In this section we describe related work about the
development and deployment of mobile native and web
context-aware applications.
Since we have created a framework to develop
proximity-based services, the state
of the art of existing frameworks, that deliver
context information to the apps will be presented.
\begin{description}
  \item[Frameworks for developing distributed
  location-based applications]
  There are frameworks to develop location-based
  applications.
  In the work presented in Krevl et al.\cite{Krevl2006},
  a framework
  was developed to allow developers to build
  location-based apps. Location information can come
  from any source, such as \gls{GPS} receivers, Bluetooth
  receivers and \gls{WiFi} receivers.
  The authors discuss some benefits and limitations
  of several technologies for getting the
  user's positioning.
  In terms of architecture, the main components
  are:
  \begin{itemize}
  \item
  Devices that are used to get location data, such as
  the ones already mentioned;
  \item The users' mobile devices;
  \item The Database Server, which is where the mapping
  between geographical coordinates and location
  information is stored;
  \item And, the Application Server, which provides web services for
  mobile clients. This server also communicates
  with the Database Server;
  \end{itemize}
  The mobile device get geographical coordinates
  from any source and send that data, in a
  \gls{SOAP}\cite{Seely:2001:SCP:560836} message,
  to the appropriate web service in the Application
  Server. This server, communicates with the Database Server
  to query the database, which sends back a response with
  location information, if there is any, for that
  particular group of geographical coordinates.
  The authors did not evaluate the system.

  This solution offers abstractions for the location
  information sources. Geographical coordinates can
  come from any source. It is a good approach for
  mobile native apps, but, it does not support web apps.
  The authors do not take into consideration
  constraints in terms of resources, such as
  lack of Internet connection and battery.
  Since most users have limited data plans for
  their smartphones and \gls{SOAP} messages can
  grow, in size, due to its \gls{XML} format,
  a more efficient message encoding could be used
  instead.
  \item[Dynamix\cite{Carlson2012}]
  is a framework to develop
  mobile native and web apps that allow them to receive
  context information, for instance, position and device's
  orientation. This framework has plugins that get
  one or more sensor's raw data and turn that into event
  objects, that contain more high-level information.
  This framework supports many kinds of context information
  and it is possible to develop more plugins to allow the
  apps to generate additional events that are not
  already supported.

  To achieve our goal, our framework could be just a
  plugin for Dynamix. The plugin would
  need to get the beacon's raw data and
  turn that into a more high-level information
  using a backend. In this framework,
  the user needs to install an app, that manages the service
  that runs in background, and needs to define some
  security policies. This could mean a big overhead since
  we are more focused on developing proximity-based applications, that do not require such complex security
  policies because, in this kind of applications, there is only need
  to access the device's sensors that could provide,
  to the applications, positioning data.
\end{description}
