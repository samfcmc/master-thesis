%!TEX root = ../dissertation.tex

\chapter{Conclusion}
\label{chapter:conclusion}
% Established some goals (which ones)
% Concept of Smart Place
% Types of location (taxonomy)
% Solution
% Implementation
% Evaluation
We have developed the Smart Places solution which offers a tool for developers to allow them to create proximity-based services and an interface for end users and owners.
Using our solution end users have access to any proximity-based service based on the concept of Smart Place, which we introduced in this work and owners can manage their Smart Places.
However, our solution is not perfect and there are limitations, which can be improved in the future.
In this chapter, we conclude this dissertation, summarizing the important conclusions of our work in section \ref{sec:conclusion_contributions} and discussing some of the limitations and future work in section \ref{sec:conclusion_future_work}.

\section{Contributions}
\label{sec:conclusion_contributions}
A Smart Place is an area with tags that mobile devices can detect and offer different possibilites to the users according to each tag.
In this dissertation, we introduced this concept, which we consider a simpler way to look at proximity-based applications.
However, these tags need to be provided by a location technology.
That technology should be compatible with most used mobile devices.
We analyzed some location technologies, using a taxonomy\cite{location}, in order to try to find a technology that would be a best fit for our concept of Smart Place.
We chose \gls{BLE} beacons using iBeacon protocol because it is compatible with any smartphone with Bluetooth version 4.0 or above.
The owners of Smart Places are responsible to place those beacons in the right place.
Users do not need any extra hardware.
They only need to download a single app and turn on the Bluetooth receivers of their mobile devices.

We have created a solution that allows developers of Smart Places to provide proximity-based services, using web technologies, such as, \gls{HTML}, \gls{CSS} and Javascript.
With this tool, any web application can react to the presence of tags placed in a given Smart Place.
This solution makes the development of Smart Places easier due to the fact that, developers do not need to handle the technology to handle tags neither the backend where the information about Smart Places and their tags is stored.
There is a mobile app for end users, anyone with a mobile device, capable of reacting to tags in a Smart Place.
Since each Smart Place is a web application they can run inside an embedded web browser in this mobile app, allowing users to have access to any Smart Place withouth the need to install a new native app for each one.

Two examples of Smart Places were created using our solution, a Smart Restaurant and a Smart Museum.
The Smart Restaurant had the goal to allow customers of a restaurant to place their orders after they take a sit using their mobile devices.
Using this solution, customers do not need to specify which table the orders belong to. Using the tag system through \gls{BLE} beacons the app automatically get the table's number and add that information when the customer places the final order.
In the Smart Museum we created the experience of a museum where visitors can have access to more information about an object that they are close to.

After the implementation phase, we evaluated the system.
We performed two sets of experiments.
The first set of experiments was performed in order to check if this distance value was reliable.
We have reached the conclusion that, if we want to rely on getting the nearest tag they need to be at least 1.5m distant from each other.
The other set of experiments was about battery consumption.
Since our mobile app has a service running in background scanning for nearby tags from time to time we need to get a good insight about the impact of that service in terms of energy.
Users will not use our solution if they run out of battery faster than if they are not using our solution.
For each experiment, we have tried two different types of data connection, using \gls{WiFi} and \gls{3G}.
Using \gls{WiFi} the power drain is almost zero.
However, using \gls{3G}, the power drain was more than 2\% of total battery, in just one hour and it consumed 3.77\%.
We have compared the power drain values in the worst scenario with one popular app, which is Facebook that also have services running in background.
We can conclude that the battery consumption is acceptable in a daily basis but there is room for more improvement.

\section{Future Work}
\label{sec:conclusion_future_work}
% No way to use other technologies for tags
% No interface for owners register beacons
% It only works for web apps... (no integration with native apps)
Our solution tries to target developers, owners and users of Smart Places.
However, our solution has some limitations that can be improved in the future.
Also, there are some features that were not implemented.

% Missing interfaces for developers and owners
Owners of Smart Places need to deploy tags and use the mobile app to configure those tags.
However, there is not any interface that they can use to register themselves as the owners of such tags.
In the development of this work we introduced the needed data manually, that is,
all the associations between tags and their owners.
There is one more interface missing to allow developers to register the Smart Places they develop.
When owners are configuring their Smart Places, they can choose from a list.
The Smart Restaurant and Smart Museum examples were introduced in the backend manually using Parse dashboard.
There is no way for developers to add a Smart Place to this list.
A future improvement could be to develop these missing interfaces.
One that would allow owners to register the ownership of tags and another for owners that would use it to register the Smart Places they developed.

% Energy consumption
One issue with our solution is the energy consumption.
As the evaluation shows, using \gls{3G}, our mobile app for end users drains more than 2\% of power in just one hour.
Comparing with the power drain of Facebook which is a popular app, running in the same period we can conclude that the power drain is acceptable but there is room for improvement.
For instance, we could store geographical coordinates for each tag.
When detecting a tag, the app could use these coordinates to fetch data about the other tags in the same area in just one request.
This way, we need less requests to the backend which can result in less power drain than the actual solution.
Also, we need to test other values smaller than 2 minutes and 30 seconds for the scanning period because it might be possible to find a smaller value resulting in an acceptable power drain.
Most recent mobile devices are \gls{4G} enabled.
This data connection mean was not tested due to limitations in terms of resources available during the development of this dissertation.

% Usability evaluation
As previously mentioned we performed experiments in order to test our Smart Places solution in terms of detecting the nearest beacon and battery consumption.
However, since we target end users and developers we need to evaluate the system in terms of usability.
In the future we could perform usability tests to check if the mobile apps are easy to use.
Besides end users there are experiments that can be done with developers to verify if we accomplish one of our goals which is making the development of Smart Places an easy process.
For instance, we could write exercises that developers would do using our \gls{API}.

% Security problem
There was no concern for security requirements. For exampple, when the Backend receives a request to associate a tag to a given beacon, it does not perform any verification, that is, it is not taken into account if the user trying to make this association is the owner of the beacon.
Any user using the mobile app for owners can associate tags to beacons belonging to any Smart Place.
In the future, the Backend needs to be improved in order to be more secure to not allow anyone to modify existing tags.

% More technologies
We picked \gls{BLE} as the technology to support tags in a given Smart Place.
However, developers might want to build Smart Places that make use of other location technologies such as, \gls{QR} codes or \gls{NFC}.
Our solution is attached to a particular technology instead of allowing to be extended in order to support more location technologies.

% Only web apps
Using our solution, only web applications can offer proximity-based services.
In one side, this is a good fit because this is what allows to have one app to access multiple Smart Places.
On the other side, there could be developers that could use our solution to add proximity-based features in their existing native mobile apps.
This could be solved having a \gls{SDK} available for the three most used mobile \glspl{OS}, iOS, Android and Windows Phone, to allow to integrate our functionality in existing apps.

% The end
Mobile devices are evolving and they are being able to get more context information and perform other tasks such as payments.
Proximity-based applications are an important type of context-aware applications since they provide services to the user, that are relevant in the right place.
However, developers need the right tools to create these applications and users need an easy way to have access to such applications.
Proximity-based applications can offer several possibilities.
For instance, companies that have online and physical stores could achieve more integration of offers such as, customers have access to exclusive discounts when they approach a given tag or a set of tags.
Another possibility is to have tags in a supermarket, which could trigger reminders in an app that customers use to store their shopping list.
For instance, the customer approaches the section where there is milk and this is on his shopping list. The app would remind him that he needs milk.
Our Smart Places solution and its evaluation can be considered a step forward to be able to use, develop and manage all these and many more possibilities.
