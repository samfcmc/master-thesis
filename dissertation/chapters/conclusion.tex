%!TEX root = ../dissertation.tex

\chapter{Conclusion}
\label{chapter:conclusion}
% Established some goals (which ones)
% Concept of Smart Place
% Types of location (taxonomy)
% Solution
% Implementation
% Evaluation
In the end, we have accomplished our goals.
We have developed the Smart Places solution which offers a tool, for developers, to allow them to create proximity-based services and an interface for end users and owners.
Using our solution, end users have access to any proximity-based service based on the concept of Smart Place, which we introduced in this work and owners can manage their Smart Places.
However, our solution is not perfect and there are limitations, which can be improved in the future.
In this chapter, we conclude this dissertation, summarizing the important conclusions of our work in section \ref{sec:conclusion_contributions} and discussing some of the limitations and future work in section \ref{sec:conclusion_future_work}.

\section{Contributions}
\label{sec:conclusion_contributions}
A Smart Place is an area with tags, that mobile devices can detect and offer different possibilites, to the users, according to each tag.
In this thesis, we introduced this concept, which we consider a simpler way to look at proximity-based applications.
However, these tags need to be provided by a location technology.
That technology should be compatible with most used smartphones.
We analyzed some location technologies, using a taxonomy\cite{location}, in order to try to find a technology that would be a best fit for our concept of Smart Place.
We chose \gls{BLE} beacons using iBeacon protocol, because it is compatible with any smartphone with Bluetooth version 4.0 or above.
The owners of Smart Places are responsible to place those beacons in the right place.
Users do not need any extra hardware.
They only need to download a single app and turn on the Bluetooth receivers of their mobile devices.

We have created a solution that allows developers, of Smart Places, that provide proximity-based services, using web technologies, such as, \gls{HTML}, \gls{CSS} and Javascript.
With this tool, any web application can react to the presence of tags placed in a given Smart Place.
This solution makes the development of Smart Places easier due to the fact that, developers do not need to handle the technology to handle tags neither the backend where the information about Smart Places and their tags is stored.
There is a mobile app for end users, anyone with a mobile device, capable of reacting to tags in a Smart Place.
Since each Smart Place is a web application they can run inside an embedded web browser in this mobile app, allowing users to have access to any Smart Place withouth the need to install a new native app for each one.
There is a mobile app for owners of Smart Places that they can use to choose which proximity-based services they want to provide and configure the necessary tags.

Two examples of Smart Places were created, using our solution, a Smart Restaurant and a Smart Museum.
In the first one we created the experience of a museum where visitors can have access to more information about an object that they are close to.
The Smart Restaurant had the goal to allow customers, of a restaurant, to place their orders, after they take a sit, using their mobile devices.
Using this solution, customers do not need to specify which table the orders belong to. Using the tag system, through \gls{BLE} beacons, the app automatically get the table's number and add that information when the customer places the final order.

After the implementation phase, we evaluated the system.
We performed two sets of experiments.
Many times, the mobile apps try to detect the nearest tag.
The library we have used in our mobile apps, have means to get the distance from the detected tags.
The first set of experiments was performed in order to check if this distance value was reliable.
We have reached the conclusion that, if we want to rely on getting the nearest tag, they need to be, at least, 1.5m distant from each other.
The other set of experiments was about battery consumption.
Since our mobile app has a service running in background scanning for nearby tags from time to time we need to get a good insight about the impact of that service in terms of energy.
Users will not use our solution if they run out of battery faster than if they are not using our solution.
For each experiment, we have tried two different types of data connection, using \gls{WiFi} and \gls{3G}.
Using \gls{WiFi} the power drain is almost zero.
However, using \gls{3G}, the power drain was above 2\%, in just one hour.

\section{Future Work}
\label{sec:conclusion_future_work}
% No way to use other technologies for tags
% No interface for owners register beacons
% It only works for web apps... (no integration with native apps)
Our solution tries to target developers, owners and users of Smart Places.
However, it is not perfect.
There are some features that were not implemented. Also, our solution has some limitations that can be improved in the future.

We picked \gls{BLE} as the technology to support tags in a given Smart Place.
Developers that want to use our solution to develop Smart Places, cannot use other technology besides this one.
This technology is a good fit because it allows to build proximity-based services where user interaction is not required, that is, users do not be aware the existence of tags.
In the Smart Museum and Smart Restaurant examples that we have created, the user does not need to open the mobile app.
The mobile device simply shows a notification with the relevant information. The user touches the notification and start interact with the Smart Place.
However, developers might want to build Smart Places that make use of other location technology, such as, \gls{QR} codes or \gls{NFC}.
Our solution is too attached to a particular technology instead of allowing to be extended in order to support more location technologies.

Owners of Smart Places need to deploy tags and use the mobile app to configure those tags.
However, there is not any interface, that they can use, to register themselves, as the owners of such tags.
In the development of this work we introduced the needed data manually, that is,
all the associations between tags and their owners.
There is one more interface missing to allow developers to register the Smart Places they develop.
When owners are configuring their Smart Places, they can choose from a list.
The Smart Restaurant and Smart Museum examples were introduced in the backend manually.
There is no way for developers to add a Smart Place to this list.
A future improvement could be these missing interfaces.
One that would allow owners to register the ownership of tags and another for owners that would use it to register the Smart Places they developed.

% Security problem
Each beacon has an owner.
However, when the Backend receives a request to associate a tag to a given beacon, it does not perform any verification, that is, it is not taken into account if the user trying to make this association is the owner of the beacon.
This introduce a security problem. Any user, using the mobile app for owners, can associate tags to beacons belonging to any Smart Place.
In the future, the Backend needs to be improved in order to be more secure to not allow anyone to modify existing tags.
%Unfortunately, anyone can use any tag as if it belongs to him/her.
%This can lead to a security problem, which is, people can associate other Smart Places to tags that already support other Smart Place.
%People that pass nearby those tags will start to get notifications of Smart Places that were not suppose to get.
%A future improvement could be a way of getting tags and register them as beloging to who ordered them.

Another limitation is the fact that, using our solution, only web applications can offer proximity-based services.
In one side, this is a good fit because this is what allows to have one app to access multiple Smart Places.
On the other side, there could be developers that could use our solution to add proximity-based features in their existing native mobile apps.
This could be solved having an \gls{SDK} available for the three most used mobile \glspl{OS}, iOS, Android and Windows Phone, to allow to integrate our functionality in existing apps.

One issue with our solution is the energy consumption.
As the evaluation shows, using \gls{3G}, our mobile app, for end users, drains more than 2\% of power in just one hour. This power drain value can be unacceptable and can demotivate the user.
Unless the owners provide \gls{WiFi} \glspl{AP} in their Smart Places, this is an issue that need to be solved in order to make this solution usable.
Most recent mobile devices are \gls{4G} enabled.
This data connection mean was not tested due to limitations in terms of resources available during the development of this master thesis.

Mobile devices are evolving and they are being able to get more context information and perform other tasks such as payments.
Proximity-based applications are an important type of context-aware applications since they provide services, to the user, that are relevant in the right place.
However, developers need the right tools to create these applications and users need an easy way to have access to such applications.
Our work in this dissertation targets not only developers but also users and owners.
A complete solution is needed because developers would not create proximity-based applications if there are no users.
Also, users would not be engaged if developers do not have the tools to create these applications.
Proximity-based applications can offer several possibilities.
For instance, we can put a tag in our garbage can that will remind us to take out the garbage.
We can have tags in our doors that would allow access only to authorized people.
Another possibility is to spread tags in a building that will provide us useful information as we navigate through it.
Our Smart Places solution and its evaluation can be considered a step forward to be able to use, develop and manage all these and many more possibilities.
