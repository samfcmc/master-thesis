%!TEX root = ../dissertation.tex

\chapter{Introduction}
\label{chapter:introduction}
% Growing use of mobile devices
% Multiple sensors
% Apps can adapt according to context
The use of mobile devices has increased in the last years.
It is now possible to use devices such as smartphones and tablets to perform tasks that previously were only possible using a desktop computer.
These devices are becoming more capable than ever before.
They have powerful \glspl{CPU} and \glspl{GPU} and are also equipped with many sensors such as light sensor \gls{GPS} and accelerometer.
Also, these devices have access to multiple data sources such as
the user's activity on social networks and calendar.
The applications (apps), that the user can install, have access to this data provided by these sensors and data sources.
Using this data, the apps can adjust settings and allow users to perform tasks according to it.
The mobile apps that use this information, which is called \emph{context}, are named context-aware applications.
For instance, an app that puts the phone in silent mode when the user is in a meeting is a context-aware app.
The user's location is a particularly useful context information because it allows apps to offer the user special functionalities when they are in a specific place.
The focus of this present work is location based applications.

\section{Motivation}
\label{sec:introduction_motivation}
% Context-aware apps
% Proximity-based
Proximity-based is a particular type of context-aware that takes into account users' location.
These apps offer different possibilities, to the users when they are in the proximity of a given point of interest.
We can find multiple examples of such apps.
Swarm\footnote{http://www.swarmapp.com} allows users to check-in in a given place or point of interest according to user's location.
For instance, if the user is in a restaurant he can use the app to perform a check-in in that restaurant and share his location on social networks.
Skout\footnote{http://www.skout.com}
is another example app that allows finding nearby users and start talking to them.
These apps are becoming popular.
Besides these examples, it is possible to automate tasks based on location.
For instance, send a message to someone when we arrive at a given location or turn on the lights when we arrive at home.
Services such as ``\acrlong{IFTTT}'' (\acrshort{IFTTT})\footnote{http://ifttt.com} allow users to make this kind of automations.
These are just examples of what is possible to do with context-aware and more specifically with location based applications.

% Develop proximity-based
To develop proximity-based apps we need to, somehow, get the device's location.
Also, we need to associate data to points of interest.
This data can be provided by the application itself or by a central server.
Developers have to develop the mobile apps and integrate in those apps the technology needed to get the proximity of the user to a point of interest.

% Offer proximity-based services
Besides development of proximity-based services another question arrises.
How can an owner of a given place offer such services without having to develop everything him/herself?
For instance, a store owner wants to advertise some promotion in the customers mobile devices when they approach the store.
How can he/she creates this promotion if there is no programming background?

% Use those services
Taking into account the previous store example, if there are multiple promotions in different stores, customers need to install one app for each store.
They should be able, by installing one app, to discover these promotions or any other proximity-based service while they are walking instead of being aware of these services and know which apps they need to install.

\section{Goals}
\label{sec:introduction_goals}
% Tool to develop proximity-based services without worrying about technologies...
In order to create proximity-based services available on mobile devices, developers need to build the mobile apps, choose the right technology for location and build the backend to maintain the data associated to each tag.
Our main goal in this dissertation, is to create a tool to develop such services%, eliminating the need to build a backend and to choose the technology.
.
With this tool, developers should be able to build the proximity-based services, without the need to develop their own backends and handle the location technology themselves.

% Introduce the concept of Smart Place
In this dissertation, we want to introduce the concept of Smart Place, which is described in further detail in the next chapter.
The idea is to have a physical place, which offers a proximity-based service due to existence of tags placed on objects that are relevant.
For instance, the example of the store can be considered a Smart Place.
The owner place some tags, inside the store, that will trigger a notification in the customers' mobile devices, such as smartphones.

% Owners and users solution
Developers are not the only category of people we aim to target.
Users should be able to access these services without having to install, on their mobile devices, one app for each service they want to use.
In the store example, users should be able to have access to the advertised promotions without the need of one app for each store.
One app should be enough, for anybody with a mobile device, such as, a smartphone, to use any proximity-based service, that is, a Smart Place where owners placed tags with useful meaning.

\section{Contributions}
\label{sec:introduction_contributions}
% Easy way of developing proximity-based services
% Adapt existing applications (web)
% Easy way of installing such services
% Easy way of using them
% Concept of Smart Place
% App for owners
% App for users
We aimed to contribute in several ways.
We created a tool to develop proximity-based applications.
There are other tools that try to solve the same problem.
We analyzed them in order to understand their advantages and limitations and built our own solution, which tries to make developers focus on their applications and not on the technology side, that is, the technology used to provide tags and make mobile apps being able to detect them.
We evaluated our solution in order to conclude how much overhead, in terms of energy consumption, is introduced by a service running in background that reacts to the presence of tags.

%Besides making easier the development of proximity-based applications, we have taken into consideration the people who manage Smart Places.
%In our solution, there is an interface, provided by a mobile app, that allows owners to manage their Smart Place. Owners can register tags and choose, from a list of available Smart Places, which ones they want to configure.
%For instance, in the store example, the owner would use this interface to choose to provide the promotions service and configure the promotion that would be advertised, when the customers are nearby each tag.

%Another contribution we can consider is the interface between, the users and the tags placed in Smart Places.
%We built a mobile app that can detect tags and offer different possibilities, according to the detected tag.
%This app allows to access instead any Smart Place, instead of users having to install one app for each Smart Place.

\section{Thesis Outline}
\label{sec:introduction_thesis_outline}
% Outline all other chapters
The rest of this dissertation is organized as follows:
\begin{description}
  \item[\chapterRef{chapter:background}]
  provides background information that is needed to understand the problem this dissertation tried to solve and the solution that is proposed;
  \item[\chapterRef{chapter:solution}]
  introduces the solution and describes its main components;
  \item[\chapterRef{chapter:implementation}]
  presents the implementation process, the solution's architecture and all the tools used to create it;
  \item[\chapterRef{chapter:evaluation}]
  shows the results from the evaluation process in order to get a good insight about the quality of the solution;
  \item[\chapterRef{chapter:conclusion}]
  concludes this dissertation and states the what can be done in the future to improve this solution.
\end{description}
