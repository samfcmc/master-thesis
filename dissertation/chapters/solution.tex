%TEX root = ../dissertation.tex

\chapter{Solution}
\label{chapter:solution}
% Overview about the solution
% Types of users
% For each one, who he is, what he does...
% What was implemented for each one
The main goal of our solution is to offer a tool
to develop proximity-based mobile applications.
There are already some tools available (presented in section \ref{sec:background_related_work}).
However, some of them are attached to a given platform, that is, developers
need to write one proximity-based application for each platform they want
to reach.
In order to circumvent this limitation, this solution allows developers
to use the same technologies that are used for any web application, such as \gls{HTML}, \gls{CSS} and Javascript.

Before going deeper in our solution, we need to take a look at three kinds of users that will be part of it.
They are the following ones:
\begin{description}
  \item[Owners:] This kind of users are the ones responsible for managing a given place that they want to turn into a Smart Place;
  \item[Developers:] The ones that develop the Smart Places;
  \item[End users:] Anyone with a mobile device that install an app to scan for nearby Smart Places.
\end{description}

Smart Places solution has a component that targets each one of the presented type of user.
Owners have a mobile app that allow them to turn the places they manage into Smart Places.
There is another mobile app to allow anyone, with a mobile device, such as a smartphone, to use the services provided by Smart Places nearby.
However, we need someone to develop these services.
That is why our solution offers an \gls{API} that developers can integrate, in their web applications, to make them react to the presence of the user.

\section{Mobile app for owners}
\label{sec:solution_mobile_app_for_owners}
% Who are the owners
% What are the main features of the app
% Workflow (with some screenshots)
As already mentioned, owners are one of the three kinds of users that our solution targets.
They manage one or more places where they want to provide some service to visitors, in their mobile devices, such as smartphones.
In order to make owners be able to offer this kind of services, an Android app, designed for them, is offered by this solution.
This app offers the folllowing features:
\begin{itemize}
  \item Get a list of all smart places available;
  \item Configure an instance of a given Smart Place;
  \item Delete an existing configuration of a given Smart Place;
  \item Update an existing instance of a given Smart Place;
\end{itemize}

After installing the app and deployed some beacons, owners can use the app to select a Smart Place and configure it, that is, create an instance of a given Smart Place and put tags on objects. Figure~\ref{fig:screenshot_ownersapp} shows three screenshots of the mobile app for owners, that illustrate the steps that the owner needs to perform in order to create an instance of a given Smart Place.
First, the app shows a list of all available Smart Places.
Then, the owner selects one and he can see a text explaining what that Smart Place is about.
Finnaly, he just needs to type a title and a message, that will appear in the users' mobile devices notifications when they are nearby.

\begin{figure}[!ht]
  \centering
    \includegraphics[width=1\textwidth, keepaspectratio]{images/screenshots/ownersapp}
    \caption{From left to right, the steps to create an instance of a given Smart Place}
    \label{fig:screenshot_ownersapp}
\end{figure}

After creating an instance of a Smart Place, the owner needs to put tags in objects, that is, associate information to the beacons that he has deployed.
This information will depend on the Smart Place.
Figure~\ref{fig:screenshot_ownersapp_configure} shows the needed steps to tag objects in order to configure an instance of a given Smart Place.
First, the owner selects the instance from the list of instances that he has created.
Then, he has access to an interface where he can manage the existing tags of that Smart Place instance.
This interface is developed by the same developers that create the Smart Place, using the \gls{API} described in section \ref{sec:solution_developers_api}.

\begin{figure}[!ht]
  \centering
    \includegraphics[width=1\textwidth, keepaspectratio]{images/screenshots/ownersapp_configure}
    \caption{From left to right, the steps to configure an instance of a given Smart Place}
    \label{fig:screenshot_ownersapp_configure}
\end{figure}

\section{Mobile app for end users}
\label{sec:solution_mobile_app_for_end_users}
% Who are end users
% What are the main features of the app
% Workflow (whith some screenshots)
Anyone that have a mobile device, such as a smartphone, can use the services provided by any Smart Place.
In our solution, there is an Android app, different from the one described in section \ref{sec:solution_mobile_app_for_owners}, that notifies the user when he is nearby any Smart Place.
When the mobile device approaches any Smart Place, the app notifies the user that he is nearby a Smart Place.
When the user touches this notifications, the app shows an embedded web browser that contains a web page that can react to nearby objects, that is, beacons with meaning to the application.

After installing the app, the first time the user opens it, the app ask the user to turn on the device's bluetooth, as it is shown in Figure~\ref{fig:screenshot_clientapp_entry}.
\begin{figure}[!ht]
  \centering
    \includegraphics[width=0.7\textwidth, keepaspectratio]{images/screenshots/clientapp_entry}
    \caption{First time the user opens the mobile app}
    \label{fig:screenshot_clientapp_entry}
\end{figure}
This app will scan for nearby beacons, periodically.
Each time a beacon is scanned it is associated to a Smart Place, the app shows a notification.
If the scanning period is small, the app can constantly notify the user. Otherwise, if this period has a big value, the user will receive less notifications.
This is why, our app allows the user to change the scan periods in background and foreground modes.

\section{Developers API}
\label{sec:solution_developers_api}
% Why javascript api
% Who
% How to use it (with some code snippets)
As already mentioned, owners configure Smart Places and end users, who are anyone with a mobile device and an app, as the one described in section \ref{sec:solution_mobile_app_for_end_users}, can interact with objects nearby.
But, who develop these Smart Places, this kind of apps that can react to nearby object with special tags?
Our solution offers a way for developers to create their Smart Places.
Also, we want to avoid the user having to install one mobile app for each Smart Place.
This is way, the app for end users,
has an embedded web browser, so they can use any Smart Place, as they would use any web application, without the need to install one more mobile app.
That is why a Javascript library is part of our solution.
This way, an existing web application can use this library and make it react to nearby objects tagged with \gls{BLE} beacons.

The library was turned into an open-source project, hosted on a github repository\footnote{http://github.com/samfcmc/smartplaces-js} and it is available, to install, using bower\footnote{http://bower.io}, which is a tool to manage frontend dependencies in web applications.
If a developer, using this tool, wants to install this library, he/she just needs to run the command, as shown in Listing~\ref{code:bower_install}

\begin{listing}[H]
  \begin{minted}{shell}
  bower install smartplaces-js --save
  \end{minted}
  \caption{Command to install smartplaces-js library using bower}
  \label{code:bower_install}
\end{listing}
Then, he/she just needs to include the library and use the available functions.
However, developers are also responsible to create the interface to configure the Smart Place, that is, the steps that owners have to follow, as described in section \ref{sec:solution_mobile_app_for_owners}, after they select the Smart Place they want.
In the part of the web app, that will be accessed by owners, developers first need to initialize the library, as shown in Listing~\ref{code:smartplaces_initialization}.
\begin{listing}[H]
  \begin{minted}{javascript}
    SmartPlaces.onInit(function(smartPlaceInstance) {
      console.log(smartPlaceInstance);
    });
  \end{minted}
  \caption{Javascript library initialization}
  \label{code:smartplaces_initialization}
\end{listing}
The library is event-based, that is, the mobile app, described in section \ref{sec:solution_mobile_app_for_end_users} emits events to the library, such as, a nearby beacon is detected, to the web application running inside the embedded web browser.
When the owner, is using the app, described in section \ref{sec:solution_mobile_app_for_owners}, there is a button that, when is touched, the app scans for nearby beacons and send an event to the javascript library.
Developers have to define the behaviour, when this event is emitted, as shown in \ref{code:smartplaces_on_beacons_scanned}.

\begin{listing}[H]
  \begin{minted}{javascript}
    SmartPlaces.onBeaconsScanned(function(beacons) {
      console.log(beacons);
    });
  \end{minted}
  \caption{Defining a callback function when beacons are scanned by the mobile app for owners}
  \label{code:smartplaces_on_beacons_scanned}
\end{listing}

The argument named ``beacons'' in callback function in \ref{code:smartplaces_on_beacons_scanned}, it is an array of \gls{JSON} objects, where each one has the following keys:
\begin{description}
  \item[uuid]: The \gls{UUID} of the beacon that was scanned;
  \item[major]: The major value, according to the ibeacon protocol;
  \item[minor]: The minor value, according to the ibeacon protocol.
\end{description}

However, there is more information about each beacon, for instance, its name and its icon \gls{URL}.
To get this extra information, from a beacon \gls{JSON} object, there is a ``getBeacon'' function, which usage is shown in Listing~\ref{code:smartplaces_get_beacon}.
Since this function makes a request to the backend, we need to pass, as an argument, an object with two keys:
\begin{description}
  \item[success]: Callback function, when the request was successfull made and we got a response with an object that, besides the keys mentioned before, \gls{UUID}, major and minor, we also got the name and icon, which is an \gls{URL} that we can use to get the image of that particular beacon;
  \item[error]: Callback function, when the request returns an error.
\end{description}

\begin{listing}[H]
  \begin{minted}{javascript}
    SmartPlaces.getBeacon(beaconScanned, {
      success: function(beacon) {
        console.log(beacon);
      },
      error: function(error) {
        console.log(error);
      }
    });
  \end{minted}
  \caption{Get beaon info from the backend}
  \label{code:smartplaces_get_beacon}
\end{listing}

After we got the beacon object with all the information, it is possible to associate a \gls{JSON} with any structure.
To do that, there is an ``associateTag'' function, which usage is illustrated in Listing~\ref{code:smartplaces_associate_tag}. We need to pass the beacon object, the object with the data that we want to associate with the beacon, and another object with success and error keys, similar to what is shown in Listing~\ref{code:smartplaces_get_beacon}.

\begin{listing}[H]
  \begin{minted}{javascript}
    SmartPlaces.associateTag(beacon, tagData, {
      success: function(tag) {
        console.log(tag);
      },
      error: function(error) {
        console.log(error);
      }
    });
  \end{minted}
  \caption{Associate a tag to a given beacon}
  \label{code:smartplaces_associate_tag}
\end{listing}

It is also possible to update an existing tag. For that, developers can use the ``updateTag'' function. Its usage is shown in Listing~\ref{code:smartplaces_update_tag}. This function require, the existing tag object and an object with success and error keys, similar to the other functions that make requests to the backend.

\begin{listing}[H]
  \begin{minted}{javascript}
    SmartPlaces.updateTag(tag, newData, {
      success: function(updatedTag) {
        console.log(updatedTag);
      },
      error: function(error) {
        console.log(error);
      }
    });
  \end{minted}
  \caption{Update data of a given tag}
  \label{code:smartplaces_update_tag}
\end{listing}

The previously mentioned functions, are available in order to make developers able to create the interfaces for owners.
For the end users, the mobile app detects nearby objects and emit an event to the web application, running inside the embedded web browser.
Developers need to define a callback function for this event.
To do that, the ``onTagFound'' can be used, as shown in Listing~\ref{code:smartplaces_tag_found}.
The tag object, which is the argument of this callback function, is the \gls{JSON} object created previously in the code illustrated in Listing~\ref{code:smartplaces_associate_tag}.

\begin{listing}[H]
  \begin{minted}{javascript}
    SmartPlaces.onTagFound(function(tag) {
      console.log(tag)
    });
  \end{minted}
  \caption{Callback for when a nearby tag is found}
  \label{code:smartplaces_tag_found}
\end{listing}

\section{Examples}
\label{sec:solution_examples}
% Why
% Introduce each one
% Explain the main features
Our solution offers a tool, for developers, to develop Smart Places, that is, proximity-based services, that users can interact with, using their mobile devices.
After defining the \gls{API}, we decided to try to use it, in order to get a good insight about its usability, before making it available for developers.
Creating these examples allowed us to understand, if this generic solution, could be applied to more than one domain.
Also, we avoided to develop the examples completely from the scratch.
Instead, we tried to integrate the \gls{JS} \gls{API}, described in section \ref{sec:solution_developers_api}, with existing applications or \glspl{API}.

Two examples were created.
The first, described in section \ref{sub:solution_smart_restaurant}, is a Smart Place, for restaurants, to allow customers place their orders without the need to wait.
The other one, is a proximity-based service for museums.
It allows users, that have the mobile app, described in section \ref{sec:solution_mobile_app_for_end_users}, having access to more information about a piece that they are close to, in a museum exhibition.
Its features are described in section \ref{sub:smart_museum}.

\subsection{Smart Restaurant}
\label{sub:solution_smart_restaurant}
The Smart Restaurant is one of the two examples of Smart Places, that were created to show the usage of the entire solution.
The main goal here is to allow restaurants' customers to place their orders, using their mobile devices, such as smartphones, without having to wait for an employee coming to them.
When customers arrive at this Smart Restaurant, a notification shows up, in their devices, with a message saying that they can place their orders using the mobile app, such as the one described in section \ref{sec:solution_mobile_app_for_end_users}.
Then, they touch the notification, and a new \gls{UI} appears.
Now, they have access to the restaurant's menu, where they can pick what they want and, in the end, place their orders.
Figure~\ref{fig:smart_restaurant_app}, shows the steps that the customer follows, to place an order, using the mobile app. We can see, from left to right, that, the app detects the table's number and after the sign in, several buttons appear, each one representing a family of products and, in the bottom, there is a button to place the order.
When the customer touches this button, there is a list of the complete order and a button to send the order.

Also, there is a backoffice, where employees and managers, have access to an \gls{UI} to manage the orders and the menu.
This backoffice was implemented in another master thesis\cite{SLOC}.
Our work here, was just to integrate our solution in this backoffice.
There was already an interface to place orders.
We changed this interface to use our Developers \gls{API}, described in section \ref{sec:solution_developers_api} to get the table's number.
We also added an option, to the backoffice's \gls{UI}, to manage the mapping between tables and beacons.

\begin{figure}[!ht]
  \centering
    \includegraphics[width=0.8\textwidth, keepaspectratio]{images/screenshots/smart_restaurant_app}
    \caption{Steps to place an order in the Smart Restaurant app}
    \label{fig:smart_restaurant_app}
\end{figure}

\subsection{Smart Museum}
\label{sub:smart_museum}
The Smart Museum is another example of a Smart Place, that is, a proximity-based service, developed using our solution.
The idea is to allow visitors, of some museum, to have access to information, about a given piece, in a given exhibition, in their mobile devices.
Visitors go to the museum and as they look at a given piece, they are notified, through the mobile app, described in section \ref{sec:solution_mobile_app_for_end_users}, that, they can get more information about that.
Then, when they touch in the notification, a screen appears with the information about that piece that they are looking at.
Figure~\ref{fig:smart_museum_app} shows a screen, in the mobile device, after the museum's visitor have touched the notification.

\begin{figure}[!ht]
  \centering
    \includegraphics[width=0.4\textwidth, keepaspectratio]{images/screenshots/smart_museum_app}
    \caption{Smart Museum app showing information}
    \label{fig:smart_museum_app}
\end{figure}

Instead of creating a fake museum, with mock data, we used real data from a real data source.
The idea was to try to emulate the real experience.
For that, we have used data from the Walters Art Museum\footnote{http://thewalters.org}, which is a public art museum, located in Mount Vernon-Belvedere, Baltimore, Maryland.
It was founded in 1934 and its collection includes more than 30 000 objects.
The way we got the data about the collections and objects is detailed in chapter \ref{chapter:implementation}, section \ref{sub:implementation_smart_museum}.
