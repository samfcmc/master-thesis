%!TEX root = ../dissertation.tex

\begin{otherlanguage}{portuguese}
\begin{abstract}
Aplicações baseadas em proximidade captam os utilizadores enquanto estes se encontram na proximidade de pontos de interesse.
Estas \textit{apps} têm vindo a tornar-se populares entre os utilizadores de dispositivos móveis.
\textit{Apps} baseadas em proximidade podem ser despoletadas em coordenadas geográficas específicas, porém, a possibilidade de poderem ser activadas com recurso a objectos com \textit{tags}, torna este tipo de aplicações mais interessantes.
Introduzimos o conceito de \textit{Smart Place} definido como um lugar fisico onde \textit{tags} são colocadas para fornecer um serviço baseado em proximidade.
Utilizadores com um dispositivo móvel capaz de detectar \textit{tags} podem utilizar o serviço fornecido quando se encontram na proximidade das referidas \textit{tags}.
Diversas tecnologias podem user usadas para a criação de \textit{tags}.

Nesta dissertação, analisamos diversas tecnologias de localização, para tentar encontrar a que melhor se adequa ao conceito de \textit{Smart Place}.
Criámos uma solução para desenvolver aplicações baseadas em proximidade, seguindo esta definição.
Também foram desenvolvidos dois exemplos de \textit{Smart Places}: o \textit{Smart Restaurant} e \textit{Smart Museum}.
As aplicações de \textit{Smart Places} permitem a qualquer um com um dispositivo móvel capaz de detectar \textit{tags} aceder a qualquer serviço baseado em proximidade, baseado no conceito de \textit{Smart Place} e criado com a nossa solução.
Foi avaliado o consumo energético desta aplicação e os resultados demonstram que a descarga de bateria pode ser aceitável tornando-a uma abordagem tecnicamente viável.

% Keywords
\begin{flushleft}

\palavrasChave{proximidade; aplicações móveis; smart place; aplicações baseadas em localização}

\end{flushleft}

\end{abstract}
\end{otherlanguage}
