%!TEX root = ../dissertation.tex

\begin{otherlanguage}{portuguese}
\begin{abstract}
Aplicações baseadas em proximidade captam os utilizadores enquanto estes se encontram na proximidade de pontos de interesse.
Estas \textit{apps} têm vindo a tornar-se populares entre os utilizadores de dispositivos móveis.
No entanto, em lugar de apenas termos pontos fixos, é possível uma abordagem mais genérica, onde uma aplicação baseada em proximidade utiliza \textit{tags}.
Estas tags podem pertencer a objectos numa posição fixa ou a objectos em movimento.
Nesta tese, introduzimos o conceito de \textit{Smart Place}, definido como, um lugar, onde \textit{tags} são colocadas, para fornecer um serviço baseado em proximidade.
Utilizadores, com um dispositivo móvel, capaz de detectar \textit{tags} podem utilizar o serviço fornecido, quando se encontram na proximidade das referidas \textit{tags}.
Diversas tecnologias podem user usadas para a criação de \textit{tags}.
Nesta tese, analisamos diversas tecnologias de localização, para tentar encontrar a que melhor se adequa ao conceito de \textit{Smart Place}.
Criámos uma solução para desenvolver aplicações baseadas em proximidade, seguindo esta definição.
Também foram desenvolvidas duas aplicações móveis.
Uma para quem gere um \textit{Smart Place} e outra para utilizadores.
A primeira, permite aos responsáveis, depois de instalar \textit{tags}, escolher quais os serviços baseados em proximidade, querem fornecer e configurar \text{tags} de acordo com os serviços fornecidos.
A aplicação para utilizadores, permite, a qualquer um com um dispositivo móvel, capaz de detectar \textit{tags}, aceder a qualquer serviço baseado em proximidade, baseado no conceito de \textit{Smart Place} e criado com a nossa solução.
Foi avaliado o consumo energético desta aplicação.
Foram também criados dois exemplos de \textit{Smart Places}.

% Keywords
\begin{flushleft}

\keywords{proximidade; aplicações móveis; smart place; localização}

\end{flushleft}

\end{abstract}
\end{otherlanguage}
