%!TEX root = ../dissertation.tex

\begin{otherlanguage}{english}
\begin{abstract}
Proximity-based applications engage users while they are in the proximity of points of interest. These apps are becoming popular among the users of mobile devices.
Proximity-based apps are triggered when the user is at specific geographic coordinates, but more interestingly, they can be triggered by tagged objects.
A Smart Place is a physical place with tags that provide a service.
Users with a mobile device capable of detecting such tags can use the provided service when they are in proximity of these tags.
Several technologies can be used to create tags.

In this dissertation, we chose bluetooth low energy beacons following the iBeacon protocol to tag the points of interest.
We have created a solution to develop proximity-based applications, based on this concept.
Also, two examples of Smart Places were built: the Smart Restaurant and Smart Museum.
The Smart Places apps allows anyone with a mobile device capable of detecting tags, to have access to any proximity-based service, based on our concept of Smart Place and created using the tool we developed.
This app was evaluated in terms of energy consumption and the results show that the battery drain can be acceptable making it a viable technical approach.

% Max: 250 words

% Keywords
\begin{flushleft}

\keywords{proximity-based; mobile apps; smart place; location based applications;
  bluetooth low energy beacons}

\end{flushleft}

\end{abstract}
\end{otherlanguage}
