%!TEX root = ../paper.tex

\section{Related Work}
\label{sec:related_work}
% Applications using beacons
% Check master project
The idea of marking physical locations with beacons has been used before
in research and practical applications.
This section is about some applications that use BLE beacons and also
about some software solutions for restaurants.

\subsection{Applications using ibeacons}
\label{sub:applications_using_ibeacons}
% BlueSentinel
\textbf{BlueSentinel}\cite{bluesentinel}: is a
occupancy detection system, for smart buildings,
that uses BLE Beacons to detect the presence of
people. The concept of a smart building
is similar to Smart Place,
due to the existence of sensors and actuators.
It is focused on the power efficiency of the
building. The idea is to optimize energy
consumption according to people's presence.
For instance, if there are no people in a given room,
the heating system can be turned off.
In this solution, the users have to install
an app, that will get the beacons' signal and
send data to a server, which will process it
and send requests to actuators in order to
perform actions to optimize the
building's power efficiency.
Unfortunately, there is a limitation
of iBeacon protocol implementation
in iOS\footnote{http://www.apple.com/ios}.
Beacons can be received, by the apps,
only when these are active. When the apps are in
background, they are waken up only to handle
enter/exit region events. To circumvent this
limitation, the authors developed custom
beacons, which advertise more than one region
in a cyclic sequence. These custom beacons
were created using an
Arduino\footnote{http://www.arduino.cc/}
and an Bluetooth USB BLE dongle.
Since this solution is a native app,
users have to install it in order
to make the smart building work to
optimize power efficiency.
Once the user starts the app, he does not
need to interact with it anymore, since it
will run in background.
\\
% Blue View
\textbf{BlueView}\cite{blueview}: is a system to help
visually impaired people to perceive some POIs.
This solution has two main components: The viewer device
and the Beacon Points (BPs). The first one is a mobile phone,
carried by the user, which is bluetooth-enabled.
The Beacon Points are just bluetooth tags instead of
BLE Beacons. The name of a POI is associated with
MAC address of the tag which it is associated to.
The steps involved in using the system are the
following: first, the viewer device will scan
for nearby BPs; then, a list of the names of
BPs is created. This list is refreshed anytime a new
BP is detected and the user is informed through auditory
feedback. The second step consists of the user, using
the viewer's device, establishing a connection with a BP
attached to an object. Finally, using audio prompt, the BP
will assist the user in locating the object.
Despite of this solution being a mobile app, installed
in the viewer's device, the authors do not have in
consideration the typical concerns of any mobile app,
such as the energy consumption.
The authors tested the application, in 2013,
using Nokia N70 as the viewer device.
This solution could be implemented using BLE Beacons
and the viewer device could be any Android, iOS or
Windows Phone smartphone.
For the audio features the smartphone's speaker or
a custom BLE Beacon with a built-in speaker could be
used.

\subsection{Software for Restaurants}
\label{subs:software_for_restaurants}
% Why software for restaurants
% Add CMS for beacons (because we talk about retailing)
\textbf{WinRest}\footnote{http://www.winrest.pt/}
WinRest is a solution for restaurants. It allows to manage payments
and customers' orders. It offers not only the software but also the
needed hardware to use it. Usually, this hardware are devices with
a touch screen. Unfortunately, customers have always to interact
with a waiter and it is the waiter that introduces all the needed data
in the system. Only waiters interact with this solution. It does not
offer any feature that allows customers to place their orders
without the need of requesting a waiter.
\\
\textbf{ZRest}\footnote{http://www.zsrest.com/}
This solution is very similar to WinRest.
But, not only the same features. They provide a feature that allows the
customer to proceed with the payment interacting with a tablet.
When it is time to pay, the waiter brings a tablet and the customer
specify how he wants to pay and if he wants the bill to come with
the tax payer number or not [not sure how to say this in english].
It does not offer any feature to allow customer to place their orders.
\\

In this section, we saw some applications that use BLE beacons and software
for restaurants. We chose this related work because our solution
brings together the BLE beacons and software for restaurants.
Looking at some applications using BLE beacons we can have good insights
about the limitations of this technology and what are the benefits and
the challenges of using it.
Software for restaurants allows us to know what are the main features
that a solution for restaurants needs to offer.
