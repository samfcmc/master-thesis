%!TEX root = ../paper.tex

\section{Conclusion and Future Work}
\label{sec:conclusion_and_future_work}
We have seen that BLE beacons can be used to develop proximity-based
mobile applications. This paper showed that we can use this technology
to create a new experience inside restaurants.
First, we saw some examples of applications using beacons and software
for restaurants. Some ideas were extracted in order to combine them.
On one side, how to use the beacons and on the other side, which
features are most used in a software solution for restaurants.
Our solution aims to reach customers, waiters and owners.
Customers can use a mobile app to place their orders.
Owners also have a mobile app to manage the restaurant's menu and to
configure the mapping between the beacons and the tables.
This mapping is what allows waiters to know which table a given order
comes from
and customers to not have to type the number of the table.
This solution also offers, to waiters, a web application were they
can see the orders.
We have established some requirements, which are the main
features that we aim to provide.
Our smart restaurant solution implements, aproximately,
88\% of all requirements. The only feature it does not offer is
the payment using the mobile app.

In the future, we are going to try to integrate our
prototype with an existing solution in order to avoid
restaurants' owners having to provide the same information
to two different applications. During the development
of this prototype we faced some challenges, such as
getting the beacons' signals and extract some useful
information from that. From these challenges, we
realized that developing proximity-based mobile
apps using this technology can become an hard task
because, we need not only to implement the mobile app
but also a backend where we map the beacons to
useful information that apps can use. Also, the users
need to download a new app for each proximity-based
service they want to use. We want to change this
by creating a framework to develop such services
in such a way that the user only needs one app
to discover them and use them and developers
can develop their proximity-based services
without the need of implementing a backend
themselves.
